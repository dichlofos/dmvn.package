\section{Юникод}
Xe\LaTeX, в отличие от \LaTeX, умеет обрабатывать исходники в кодировке UTF-8, без участия пакета {\it inputenc}, что дает огромные преимущества.
Например, я могу написать
$$
α±β = ¼
 $$

и это будет красиво выглядеть на печати. Но на вашей клавиатуре
символа $±$ скорее всего нет.  Поэтому мы воспользуемся аналогом
ComposeKey для Windows - Unichars.  По удобству с этим проектом
(\url{https://github.com/kragen/xcompose}) для XServer не сравнится,
но все равно неплохо, и, похоже, лучшее в своем роде.  Итак, скачиваем
программу с \par\url{http://unichars.sourceforge.net/en/frame.html}.

Скачанный установщик запускаем и соглашаемся на все параметры по
умолчанию, однако не забудьте убедится, что галочка {\it autorun}
установлена.  Директория установки по умолчанию -- папка загрузки, что
может быть или не быть удобным.  Теперь у вас в системном трее
появилась бирюзовая иконка. Если вы нажмете и отпустите(Ctrl),то она
станет красной. Это означает, что теперь UniChars ждет комбинацию
символов. Если ещё раз нажать и отпустить (Ctrl), то она станет
желтой.  Теперь, например, можно нажать {\bf g}, потом {\bf r}, то
получим букву ρ. Внимание! Раскладка должна быть в этот момент
английской.  Если вы передумали, можно нажать {\bf Esc}, и UniChars
перестанет ждать комбинацию. Обозначим операцию нажать и отпустить
(Ctrl) MultiKey. Вот список самых часто используемых комбинаций. Все
помнить незачем, они используют довольно интуитивную мнемонику, да и
UniChars предлагает всплывающую подсказку, хотя пожалуй и мелковатую.
\begin{itemize}
  \item MultiKey + MulitKey + Arrow + Esc= → ↑↓←→
  \item MultiKey + MultiKey + g + [a..zA..Z] + Esc = [α..ωΑΩ]   
\end{itemize}
Полный список можно посмотреть и отредактировать, нажав правой кнопкой на иконку в трее → Main Window. Ещё можно поменять (Ctrl) на (Win), 
например…
