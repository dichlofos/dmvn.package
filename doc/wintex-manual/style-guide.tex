\section{Рекомендации по стилю.}
В принципе, абсолютно все рекомендации можно сократить до одной --
прочитайте исходники dmvn-пакета. Но все же…
\begin{itemize}
\item Используйте \verb'\dmvntitle' и \verb'\dmvntrail' 
  вместо самописных титульных листов. 
\item Используйте макросы LaTeX-стиля по максимуму. Используйте то,
  что макросы можно объявлять не только в преамбуле, но и в самом
  документе, причем они получаются локальными для ближайшей группы.
\item Широкий монитор -- это круто. Но не повод делать строчки длиной
  в 160 символов. Обычно рекомендуется ограничиваться 79 символами.
  Этот документ успешно и читабельно набран с ширинов в 70 символов.
\item Не используйте форсированный перевод строки \verb'\\' -- не
  оберетеся проблем с \{under,over\}full. Используйте создание нового
  параграфа, причем желательно пустой строкой, а не \verb'\par'.
\item Однобуквенные слова(предлоги и местоимение {\tt я}) пишутся через неразрывный пробел(тильду).

\end{itemize}
